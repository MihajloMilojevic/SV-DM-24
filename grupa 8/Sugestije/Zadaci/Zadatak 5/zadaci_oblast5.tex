
\documentclass[a4paper,12pt]{article}
\usepackage[utf8]{inputenc}
\usepackage[serbian]{babel}
\usepackage{amsmath,amsfonts,amssymb}

\begin{document}

\section*{Zadaci oblast 5}

\subsection*{Zadatak 1}
Rešiti rekurentnu relaciju:
\[
a_n^2 - 2a_{n-1} = 0, \quad za \; n \geq 1,
\]
ako je $a_0 = 2$.

Rešenje: Kako je $a_n^2 = 2a_{n-1}$, logaritmovanjem jednačine dobijamo:
\[
2 \log_2 a_n = \log_2 2 + \log_2 a_{n-1} = 1 + \log_2 a_{n-1}.
\]
Smenom $b_n = \log_2 a_n$ se dobija nehomogena rekurentna relacija:
\[
2b_n = 1 + b_{n-1},
\]
koja zadovoljava početni uslov $b_0 = \log_2 2 = 1$. Rešenje homogenog dela je $h_n = A \left( \frac{1}{2} \right)^n$, a partikularno rešenje tražimo kao $p_n = C$. Upravljanjem $p_n$ u nehomogenu jednačinu dobijamo $C = 1$, a zbog početnog uslova važi $1 = b_0 = A (1) + 1$, pa je $A = 0$. Dobijamo:
\[
b_n = 1,
\]
pa je $a_n = 2^1 = 2$ za sve $n \geq 1$.

\subsection*{Zadatak 2}
Rešiti sistem rekurentnih relacija:
\begin{align*}
a_{n+1} &= 3b_n, \\
b_{n+1} &= 2a_n - b_n,
\end{align*}
ako je $a_0 = 1$ i $b_0 = 0$.

Rešenje: Kada u drugu jednačinu uvrstimo da je $a_n = 3b_{n-1}$ dobijamo rekurentnu relaciju $b_{n+1} = 6b_{n-1} - b_n$. Njena karakteristična jednačina $t^2 + t - 6 = 0$ ima korene $t_1 = 3$ i $t_2 = -2$, te je $b_n = A(3^n) + B(-2)^n$. Za rešavanje dobijene rekurentne relacije drugog reda trebaju nam dva početna uslova, $b_0$ i $b_1$. Kako je $b_1 = 2a_0 - b_0 = 2$, iz druge jednačine, konstante $A$ i $B$ dobijamo rešenjem sistema:
\[
A + B = 0, \\
-3A + 2B = 2.
\]
Sada je $A = -\frac{2}{5}$ i $B = \frac{2}{5}$, te je $b_n = -\frac{2}{5}(-2)^n + \frac{2}{5}(3^n)$. Opšti član niza $a_n$ na kraju dobijamo na sledeći način:
\[
a_n = 3b_{n-1} = \frac{2}{5}(-3)^n + \frac{3}{5}(2^n).
\]

\subsection*{Zadatak 3}
Rešite nehomogenu rekurentnu relaciju:
\[
b_n = 3b_{n-1} + 2^n
\]
sa početnim uslovom \(b_0 = 2\).

Rešenje:
1. Prvo rešavamo homogeni deo:
\[
b_n^H = 3b_{n-1}^H
\]
Karakteristična jednačina je \(r = 3\), pa je homogeno rešenje oblika:
\[
b_n^H = C \cdot 3^n
\]

2. Za partikularno rešenje pretpostavljamo oblik \(b_n^P = A \cdot 2^n\), jer je desna strana \(2^n\).  
Uvrstimo u jednačinu:
\[
A \cdot 2^n = 3 \cdot A \cdot 2^{n-1} + 2^n
\]
Deljenjem sa \(2^n\) dobijamo:
\[
A = \frac{2}{1} = 2
\]
Dakle, \(b_n^P = 2 \cdot 2^n\).

3. Opšte rešenje je:
\[
b_n = b_n^H + b_n^P = C \cdot 3^n + 2 \cdot 2^n
\]

4. Koristeći početni uslov \(b_0 = 2\), dobijamo:
\[
b_0 = C \cdot 3^0 + 2 \cdot 2^0 = C + 2 \implies C = 0
\]

Tako da je rešenje:
\[
b_n = 2 \cdot 2^n
\]

\end{document}
