\documentclass{article}

\begin{document}

\subsubsection*{Zadatak 7}
U grupi od 20 šahista nalazi se 5 velemajstora. Na koliko načina se mogu formirati dve ekipe od po 10 šahista tako da u prvoj ekipi bude dva velemajstora, a u drugoj tri?

\noindent \textbf{Rešenje:}
Za prvu ekipu biramo 2 velemajstora i 8 običnih šahista, što se može uraditi na 
\[
C(5, 2) \times C(15, 8)
\]
načina. Preostali šahisti automatski idu u drugu ekipu. Ukupan broj načina je
\[
C(5, 2) \times C(15, 8) = 10 \times 6435 = 64350.
\]
Dakle, postoji 64,350 načina.

\end{document}